

\section{Problem 1: Problem Statement Title}
\textbf{Problem:} Write the problem statement here. You can describe the problem in detail, such as equations, diagrams, or concepts involved.

\vspace{1em} % Adds space between problem and solution

\textbf{Solution:} Here, write your solution to the problem. This can include the following steps:
\begin{itemize}
    \item Step 1: Description or calculation.
    \item Step 2: Description or calculation.
    \item Step 3: Final solution or conclusion.
\end{itemize}

You can use mathematical formatting, such as:
\[
f(x) = \int_{0}^{\infty} e^{-x^2} \, dx
\]
or numbered equations like:
\begin{equation}
    E = mc^2
\end{equation}

\newpage

\section*{Problem 2: Another Problem Title}
\textbf{Problem:} Another problem statement goes here.

\vspace{1em}

\textbf{Solution:} Another solution explanation with steps:
\begin{itemize}
    \item Step 1: Explanation.
    \item Step 2: Further solution.
\end{itemize}

You can also include diagrams or plots, for example:
\begin{figure}[h]
    \centering
    \includegraphics[width=0.5\textwidth]{example-image} % Change to the path of your image
    \caption{A sample diagram.}
    \label{fig:sample-diagram}
\end{figure}

\newpage

\section{Fazendo referncias bibliograficas}

Para fazer referencias bibliograficas em LaTex, voce deve usar o comando \cite{dirac} para fazer uma citação à algum artigo mencionado no arquivo \texttt{Referencias.bib}.

\cite{website:ArduinoLabview}
\cite{website:OPENSOURCESW}
\cite{website:OPENSOURCEHW}


\section{Hotkeys}

\begin{table}[h]
    \centering
    \begin{tabular}{| l | l |}
        \hline
        \textbf{Comando} & \textbf{Operação} \\
        \hline
        CMD + B & Ativa negrita em seleção \\
        CMD + I & Ativa italico em seleção \\
        Ctrl + U & Ativa uppercase em seleção \\
        Shift + Ctrl + U & Desativa uppercase em seleção \\
        \hline
    \end{tabular}
    \caption{Comandos interessantes}
    \label{tab:my_label}
\end{table}